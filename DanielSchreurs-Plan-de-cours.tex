\documentclass[11pt,a4paper]{article}
%\usepackage{natbib}
\usepackage[natbib=true,style=apa]{biblatex}
\addbibresource{citations.bib}
\usepackage[french]{babel}
\usepackage{textcomp}
\usepackage{csquotes}
\usepackage{setspace}
\usepackage{fancybox}
\usepackage{fancyhdr}
\setlength {\marginparwidth }{2cm}
\usepackage{todonotes}
\usepackage{lipsum}
\usepackage{amsmath}
\usepackage{amsfonts}
\usepackage{amssymb}
\usepackage{pifont}% http://ctan.org/pkg/pifont
\newcommand{\cmark}{\ding{51}}%
\newcommand{\xmark}{\ding{55}}%
\usepackage{bookmark}
\usepackage{mathtools}
\usepackage{scalerel}
\usepackage{multirow}
\usepackage{diagbox, eqparbox, hhline}
% https://tex.stackexchange.com/questions/150634/how-to-force-a-text-to-appear-after-a-table
\usepackage{placeins}
\usepackage{graphicx}
\usepackage{booktabs}
\usepackage{lscape}
\graphicspath{{Img/}}
\usepackage{float}
\usepackage{titlesec}%remove chapter N
\usepackage{soul} % Texte surligné
\usepackage[left=2cm,right=2cm,top=2cm,bottom=3cm]{geometry}
\setlength{\parindent}{0cm}
\usepackage[framemethod=tikz]{mdframed} %highlight an entire paragraph
\usepackage{framed}
\usepackage{adjustbox}
\usepackage{array}
\usepackage{glossaries}


\usepackage{caption}



\usepackage{comment}


\usepackage{color,soul}
\usepackage{marginnote}
\hypersetup{
    colorlinks,
    citecolor=black,
    filecolor=black,
    linkcolor=black,
    urlcolor=blue
}

\usepackage{listings}
\lstset{
    numbers=left,
    numberstyle=\sffamily\tiny,
    escapeinside={<@}{@>}
}

\usepackage{pdfpages}

\usepackage{hyperref}

\setcounter{secnumdepth}{3}
\setcounter{tocdepth}{3}

\makeatother



\let\labelitemi\labelitemii



\setlength{\doublerulesep}{2.5pt}
\frenchbsetup{ItemLabeli=\textbullet}
\begin{document}
\newcommand{\JMUTitle}[9]{
    \thispagestyle{empty}
    \vspace*{\stretch{1}}
    {\parindent0cm
        \rule{\linewidth}{.7ex}}
    \begin{flushright}
        \vspace*{\stretch{1}}
        \bfseries\Huge
        #1\\
        \vspace*{\stretch{1}}
        \bfseries\large
        #4\\
        \vspace*{\stretch{1}}
        \bfseries\large
        #9
    \end{flushright}
    \rule{\linewidth}{.7ex}

    \vspace*{\stretch{1}}
    \begin{center}
        \vspace*{\stretch{1}}
        \Large #3 \\

        \vspace*{\stretch{2}}
        \large IFRES. Formasup/CAPAES\\
        \vspace*{\stretch{1}}
        \large   #8 \\[1mm]
        %\vspace*{\stretch{1}}
        \large Année académique: #5 - #6
        %large W{\"u}rzburg, den #6
    \end{center}
}
\JMUTitle
{Plan de cours : Multimédia}                                % Titel der Arbeit
{Kurztitel der Arbeit}                            % Muss in die Kopfzeile passen
{Plan de cours à rédiger dans le cadre du PESU0016}       % Art der Arbeit
{Schreurs, Daniel }                              % Vor- und Nachname des Autors
{2022}                                      % Tag der Anemeldung 
{2023}                                      % Tag der Abgabe
{Bachelor/Master Wirtschaftsinformatik}           % Studiengang
{Pascal Detroz, Dominique Verpoorten, Catherine Delfosse et Françoise Jérôme}                       % Name des Betreuers -- Hier sollte *immer* Prof. Winkelmann stehen
{Haute École de la Province de Liège}                                        % Matrikelnummer 
\clearpage
\tableofcontents
\addtocontents{toc}{\protect\thispagestyle{empty}}
\pagenumbering{gobble}


\clearpage
\pagenumbering{arabic}
\section{Informations de base}

\begin{table}[H]
    \begin{tabular}{|l|l|}
        \hline
        Cycle                                        & 1                                  \\ \hline
        Niveau du cadre francophone de certification & 6                                  \\ \hline
        Code                                         & GRA-1-048 2.2.1                    \\ \hline
        Crédits ECTS                                 & 6                                  \\ \hline
        Volume horaire (h/an)                        & 60                                 \\ \hline
        Période                                      & Quadrimestre~2                     \\ \hline
        Implantation(s)                              & TECHNIQUE — Seraing                \\ \hline
        Unité                                        & Orientation                        \\ \hline
        Responsable de la fiche                      & SCHREURS Daniel                    \\ \hline
        Pondération                                  & 60                                 \\ \hline
        Composition de l'unité d'enseignement        & Mutimédia — TP                     \\ \hline
        Prérequis                                    & /                                  \\ \hline
        Corequis                                     & Développement Côté Client (DCC)    \\ \hline
        Intervenants                                 & Maitre-assistant~: SCHREURS Daniel \\ \hline
        Contact                                      & {\ul daniel.schreurs@hepl.be}      \\ \hline
    \end{tabular}
\end{table}

\section{Description du cours}

Au premier quadrimestre, vous avez suivi un cours de «~Développement Côté Client (DCC)~». Vous y avez appris les bases essentielles de la programmation en JavaScript. Nous allons maintenant poursuivre l'apprentissage de ce langage pour aller bien plus loin jusqu'à la réalisation de jeux 2D dans un navigateur. Cet apprentissage est fondamental pour votre futur métier. JavaScript est devenu un langage de programmation incroyablement populaire\footnote{Selon le rapport de l'institut de recherche \href{https://insights.stackoverflow.com/survey}{Stack Overflow}.} et polyvalent qui est utilisé dans de nombreux domaines différents. C'est un langage de programmation Web de premier plan. Si vous voulez créer des sites Web interactifs ou des applications Web, il est essentiel de connaitre JavaScript. De plus, c'est un langage de programmation universel, puisqu'il est utilisé non seulement pour le développement Web, mais aussi pour la création d'applications mobiles, de jeux, d'applications de bureau et même de logiciels embarqués. C'est un langage de programmation en demande sur le marché de l'emploi\footnote{Selon le classement des langages de programmation publié par le site d'emploi \href{https://www.indeed.com/jobtrends/javascript.html}{Indeed} et le rapport de l'entreprise de recrutement de développeurs technologiques \href{https://www.dice.com}{Dice}.} et cela ne devrait pas changer de sitôt\footnote{Selon le rapport de l'entreprise de recrutement de développeurs technologiques \href{https://hired.com/}{Hired}.}. Si vous voulez augmenter vos chances de trouver un emploi dans l'industrie de la technologie, la maitrise de JavaScript est un excellent investissement.\\

J'ai choisi d'articuler mon cours autour de jeux, car ils sont un moyen amusant et motivant de relever des nouveaux défis en programmation. Ils me permettent d'illustrer les concepts de la programmation de manière concrète et interactive. Ces concepts dont vous avez besoin pour votre futur métier. Ce sera très gratifiant pour vous, car vous verrez très rapidement des résultats et vous pourrez même montrer vos réalisations personnelles à un futur employeur. Les jeux sont aussi un excellent moyen de se familiariser avec les différentes étapes du développement logiciel, comme la planification, la conception, la mise en œuvre et le débogage, etc.

\clearpage
\section{Ma philosophie de l'apprentissage}

La première langue que j'ai apprise (et que je parle à la maison) est l'allemand. Je n'ai donc pas toujours eu une scolarité aisée. Si aujourd'hui je suis passé outre cette difficulté et que je suis même passé de l'autre côté du banc c'est aussi parce que j'ai eu la chance de rencontrer, durant mon parcours, des enseignants qui ont cru en moi et qui ont su me motiver. Je souhaite donc, à mon tour, aussi donner une chance au plus grand nombre en vous motivant.\\

Quand je donne cours, j’essaye que chacun ressente le caractère atteignable du cours, particulièrement au début. Par exemple, je prends beaucoup de plaisir, à présenter les projets de vos prédécesseurs, je le fais, car il s’agit là d’une preuve que d’autres y parviennent. Et donc… pourquoi pas vous~? De plus, je fais systématiquement, en début de quadrimestre, un petit test formatif. Je souhaite ainsi identifier ce que vous maitrisez déjà en vue d’adapter les cours en fonction de vos acquis. Cela me permet aussi de vous rediriger, au besoin, vers d'autres ressources spécifiques de manière individuelle. Enfin, j'organise la matière en plaçant stratégiquement la difficulté de sorte qu'elle soit accessible au plus grand nombre le plus longtemps possible. Le tout dans un contexte plutôt libre et autonome afin d'installer un climat de classe motivant tout en évitant la carotte et le bâton. Je cherche plutôt à vous challenger sur des nouveaux défis.\\

Même si les concepts sous-jacents restent valables, les technologies que j'ai étudiées au début de mon parcours supérieur sont aujourd'hui obsolètes. Non pas parce qu'on m'a enseigné des outils obsolètes, mais parce que l'évolution des nouvelles technologies, dans le domaine de l'information, est rapide. En quelques années à peine, il peut y avoir des évolutions significatives. J’essaye donc de répondre à cette réalité en favorisant votre autonomie. Je cherche à vous donner les clés pour comprendre les textes techniques qui vous permettront d’aborder d’autres nouvelles technologies. Il s’agira donc beaucoup «~d’apprendre à apprendre~». Nous aurons régulièrement l’occasion d’analyser des problèmes et d’y apporter des solutions concrètes individuellement ou collectivement. J’utilise donc l'exploration pour introduire les nouveaux concepts et l’apprentissage par projets pour consolider vos connaissances. Je privilégie les activités d’apprentissage\footnote{J'encourage la prise de parole en classe, en veillant à ce que chacun se sente en sécurité et à l'aise.} en petits groupes (moins de 20) afin de favoriser votre participation et votre sentiment d’inclusion. J’essaye ainsi de vous donner un cadre moins intimidant.


Enfin, j'offre un soutien et une ouverture aux élèves qui se sentent marginalisés ou discriminés, en leur offrant un espace sécuritaire et en leur proposant des ressources pour obtenir de l'aide et de l'assistance.

\subsection{Justifications}

J’ai trouvé important de rédiger cette section, car je souhaite que mes étudiants comprennent mieux mon approche de l'enseignement et sachent ce à quoi ils peuvent s'attendre dans mes cours. J'espère que cela leur donne un cadre de référence pour leur propre approche de l'apprentissage. D’ailleurs, selon \citet{barkley2014collaborative} les étudiants qui réfléchissent sur leur propre philosophie d'apprentissage sont mieux équipés pour adapter leur apprentissage aux différentes situations et environnements d'apprentissage.
J'ai délibérément choisi de partir d'expériences personnelles, en vue de me présenter comme un prof accessible/disponible tout en évitant de présenter des valeurs creuses et impersonnelles. Dans le premier paragraphe de ma philosophie d'apprentissage, je vois une occasion de briser ce mythe du prof intouchable sur son piédestal qui transmet un savoir. Je suis donc parti d'expériences personnelles pour arriver aux valeurs. Ensuite, j'illustre avec des cas concrets comment je mets en place ces valeurs dans mon enseignement.

\clearpage
\section{Prérequis et corequis}

Ce cours s’inscrit dans la continuité du cours de «~Développement Côté Client~», car il se donne au premier quadrimestre et que vous y avez acquis les bases de la programmation en JavaScript. Nous allons maintenant nous servir de ces concepts pour aller plus loin et construire des interfaces multimédias riches. Le cours de «~Développement Côté Client~» devient ainsi le corequis de ce cours.
Au premier cours, j'organise un petit test formatif qui permet de mesurer votre maitrise en JavaScript. Ainsi je pourrai revenir vers vous individuellement pour vous orienter vers des ressources, si vous n’avez pas compris un concept. Si vous éprouvez des difficultés en JavaScript, je vous encourage d’une part à refaire les exercices du cours\footnote{Je vous rappelle que les correctifs des exercices sont également disponibles depuis la branche «~complete~».} avec les vidéos explicatives de la chaine «~\href{https://www.youtube.com/@coursdeweb}{coursdeweb}~»\footnote{youtube.com/\@coursdeweb}. D’autre part à suivre la petite formation en ligne «~\href{https://javascript30.com}{JavaScript30}~»\footnote{javascript30.com} de \href{https://wesbos.com}{Wes Bos}\footnote{wesbos.com}.

\begin{figure}[H]
    \begin{center}
        \includegraphics[width=\textwidth]{figures/corequis.eps}
        \caption{Illustration du corequis}
        \label{Fig:GQM}
    \end{center}
\end{figure}
\clearpage

\section{Contenus}

Voici dans l'ordre les différents thèmes que nous aborderons ensemble en classe. Les différentes séances de cours sont organisées avec une complexité croissante. Cela vous permet de mieux comprendre et de retenir les informations. Si le cours est trop complexe au début, vous pourriez vous sentir dépassés et perdre rapidement le fil de ce qui est enseigné. En organisant la complexité de manière progressive, vous avez le temps de comprendre les concepts de base avant de passer aux parties plus sophistiquées. Cela rend le cours plus agréable et engageant. Si le cours est trop difficile, vous pourriez perdre la motivation et l'intérêt.


\begin{enumerate}
    \item Test formatif en JavaScript.
    \item Correction et rappels des concepts de base JavaScript utilisés dans le cadre de ce cours.
    \item Utilisation d’un framework pour compiler les fichiers sources.
    \item Réalisation d'animations 2D simples avec JavaScript.
    \item Réalisation d’un outil qui permet de générer un logo à partir de paramètres encodé par l’utilisateur.
    \item Introduction à l’API de Canvas.
    \item Révision de quelques concepts mathématiques essentiels pour animer des formes. (Radian, degré, périmètres, Sin, Cos, etc.)
    \item Mise en place d’une boucle d’animation. Déplacer aléatoirement et à vitesse constante, des formes dans un canvas.
    \item Déplacer plusieurs formes avec la détection du survol de la souris.
    \item Détecter et interagir avec les évènements émis par l'utilisateur. Clic, survol, clavier, etc..
    \item Utilisez l’API Canvas pour appliquer des traitements sur des images bitmap.
    \item Déplacer des formes avec des images dans un canvas.
    \item Simuler de la neige, de la pluie sous l'effet du vent.
    \item Dessiner et animer le décor d’un jeu 2D avec une sprite sheet.
    \item Réalisation d’un premier jeu complet Flappybird.
    \item Réalisation d’un deuxième jeu complet Asteroids.
    \item Réalisation d’un examen formatif des années précédentes.
    \item Correction de l'examen formatif.
\end{enumerate}

\clearpage
\subsection{Justifications}

J'ai cherché à calibrer la complexité du cours pour permettre aux étudiants de mieux comprendre et de retenir les informations. Une étude menée par \citet{mayer2003nine} a montré que lorsque les étudiants apprennent de nouvelles informations de manière progressive, ils ont de meilleures performances que lorsqu'ils apprennent de manière non progressive. Cela s'explique par le fait que lorsque les étudiants apprennent de manière progressive, ils ont l'occasion de mettre en relation les nouvelles informations avec ce qu'ils savent déjà, ce qui peut les aider à mieux comprendre et retenir. \citet{gagne1974principles} rejoint cette idée. Si les étudiants sont constamment confrontés à des concepts difficiles, ils peuvent avoir du mal à suivre et à acquérir de nouvelles compétences de manière efficace. En organisant la complexité de manière progressive, les étudiants ont la possibilité de s'exercer et de mettre en pratique ce qu'ils ont appris avant de passer aux concepts plus difficiles.\\

Enfin, cela peut rendre le cours plus agréable et engageant pour les étudiants. Selon \citet{keller1987development}, la motivation des étudiants est influencée par leur perception de l'intérêt et de la pertinence du cours, ainsi que par leur perception de leurs propres compétences et de leur progression. Si le cours est trop difficile, les étudiants peuvent perdre la motivation et l'intérêt. En organisant la complexité de manière progressive, les étudiants peuvent sentir qu'ils progressent et atteignent des étapes importantes, ce qui peut renforcer leur engagement et leur motivation.

\clearpage
\section{Visées d’apprentissage}

Ce cours de «~multimédia~» vise à~:
\begin{enumerate}
    \item Développer votre aptitude à programmer en JavaScript. Plus exactement, à partir d'un énoncé, exprimé en français, proposer un programme en JavaScript \underline{efficace} dans \underline{un navigateur} qui respecte \underline{nos critères des qualités}. Ce qui implique de :
          \begin{enumerate}
              \item Connaitre les \underline{concepts} fondamentaux\footnote{Il s'agit là d'une dizaine de concepts qui reviennent tout le temps et qui seront identifiés en tant que tels.} utilisés dans la programmation avec JavaScript~;
              \item Comprendre ces concepts~;
              \item Identifier les concepts dont vous aurez besoin pour un cas pratique donné~;
              \item Savoir combiner différents concepts~;
              \item Savoir utiliser \underline{les outils de développement} vus en cours.\\
          \end{enumerate}
\end{enumerate}


Les mots ont de l'importance et je souhaite qu'on s'entende sur ceux-ci :

\begin{itemize}
    \item \underline{Concepts}~: c'est une «~idée~» ou une «~notion~» qui est importante et dont on a besoin pour programmer en JavaScript. Elle est utilisée pour comprendre et résumer quelque chose de manière générale~;
    \item \underline{Efficace}~: c'est à dire qui a été conçue de sorte à pouvoir fonctionner sans pour autant consommer \emph{inutilement} des ressources~;
    \item \underline{Nos critères des qualités}~: sont définies \href{https://github.com/tecg-dcc/dcc-guidelines}{ici}\footnote{github.com/tecg-dcc/dcc-guidelines} dans notre guide des bonnes pratiques. Ce sont ces règles que nous appliquons systématiquement en classes~;
    \item \underline{Un navigateur}~: je ne précise pas lequel, car vous devez utiliser les techniques vues en classe qui sont supportés par tous les navigateurs~;
    \item \underline{Les outils de développement}~: il s'agit là simplement des logiciels que nous utilisons en classes. Par exemple, PhpStorm, Git, Brave, etc.
\end{itemize}

\clearpage
\subsection{Justifications}

Je trouve cette partie particulièrement importante pour les étudiants. Finalement, c'est ici que je fixe mes attentes. J'ai donc essayé de les rendre aussi compréhensibles que possible à tel point que j'ai demandé à 2 de mes cousins, qui sont en 6\up{e} secondaire, de lire cette partie et de m’expliquer ce qu’ils comprennent. J'ai bien conscience du manque de rigueur et des lacunes que présente cette approche. L'intention étant ici, dans le temps qu'il m'est imparti, de tester la compréhension des visées. Bien entendu, il serait souhaitable de le faire avec \emph{mes} étudiants et un nombre plus important de sujets. Il n'empêche qu'à l’issue de cet échange, j'ai adapté l'écriture des visées d'apprentissage. Mes cousins avaient du mal avec certains mots. Je me suis donc permis de les définir pour que mes étudiants et moi ayons la même représentation derrière ces mots. Je n'ai pas mis en italique les verbes d'action de la taxonomie de Bloom afin d'éviter de distraire mes étudiants. Pour mes cousins, la mise en évidence de ces verbes ne les aidait pas à comprendre.\\

Ces précisions sur la \emph{forme} apportées, j'en viens au \emph{fond}. J'ai choisi de travailler avec une \emph{approche par objectifs (APO)}. Initialement, j’étais parti sur l’idée d’exprimer une seule compétence de très haut niveau, mais ce n'est pas ce qui correspond le mieux à la réalité. Je cherche à entrainer leur capacité à identifier les concepts dont ils ont besoin à partir d'une exigence formulée en français. Après avoir identifié ces concepts, ils doivent pouvoir les mettre en pratique. Ce sont toujours les mêmes concepts qui reviennent même si la combinaison peut être unique puisqu'elle dépend de l'exigence. Ils ne sont donc pas toujours dans des situations parfaitement nouvelles. Pour ce cours, je ne suis pas vraiment dans une approche par compétences, c’est plutôt une approche par objectifs. Je suis conscient des limites de cette approche. Par exemple, j'ai du mal à déterminer le niveau de spécificité. Par exemple, mes objectifs spécifiques pourraient encore être détaillés en donnant davantage de précision sur le contexte et sur la mesure de la performance, mais où placer le curseur ? Aussi les objectifs peuvent être artificiellement simples et par définition \emph{peu intégrés}. Je compense cela en essayant de formuler des objectifs de haut niveau qui justement ne se limite pas à la description d'une tâche simple, mais une tache intégrative et authentique. J'utilise la taxonomie de Bloom revue par \citet{anderson2001taxonomy} pour décrire mes objectifs. Pour deux raisons, elle est facile à comprendre pour mes étudiants et mes objectifs d'apprentissages sont du domaine cognitif.

Dans cette liste, je mets en relation les niveaux de la taxonomie de Bloom avec mes objectifs spécifiques. Comme on peut le voir, tous les niveaux sont exploités sauf l’\emph{évaluation}, ce niveau qui vise à porter un jugement critique. Dans le cadre de ce cours, ce n'est pas ce que nous travaillons.
\begin{enumerate}
    \item \underline{Se rappeler} : Connaitre les concepts...
    \item \underline{Comprendre} : Comprendre les concepts...
    \item \underline{Appliquer} : Savoir utiliser les outils\footnote{Les outils sont les logiciels qu'ils utilisent pour programmer.} de développement...
    \item \underline{Analyser} : Identifier les concepts...
    \item \underline{Évaluer} :
    \item \underline{Créer} : Savoir combiner différents concepts... en vue de «~créer~»
\end{enumerate}

Au moment de rédiger ces lignes, ce cours est évalué de manière isolée. Mais il faut savoir qu'il s'agit là d'un des 3 langages fondamentaux du web. Nous (professeurs du bachelier) avons déjà évoqué plusieurs fois l'idée de proposer une seule tache intégrée avec ces deux autres cours. L'objectif serait alors de formuler des objectifs transversaux et de les évaluer de manière intégrée. À ce stade, nous sommes toujours en phase de réflexion, mais c'est vers cela que nous aimerions aller. Avec \emph{une approche programme}.

\clearpage
\section{Méthodes d’enseignement et activités d’apprentissage}
\label{methodes_d_enseignement_et_activites_d_apprentissage}
Vous avez choisi un bachelier professionnalisant, qui cherche donc à vous préparer, au mieux, au monde professionnel. C’est pourquoi j’ai choisi d’articuler le développement de vos compétences autour de cas réels issu de jeux vidéos. Ils permettent de comprendre concrètement et de manière interactive les concepts de la programmation dont vous avez besoin dans votre futur métier. Le cours se donne au deuxième quadrimestre, une fois par semaine à raison de 4~heures. Voici les types d'activités que vous allez vivre :
\begin{itemize}
    \item Je vous expliquerai de manière claire et concise, avec beaucoup d’exemples concrets, les concepts dont vous avez besoin pour programmer en JavaScript.
    \item Il y aura aussi des moments d'échanges, je vais vous faire réagir et débattre sur la matière que nous voyons. Je m'engage à organiser ces moments et à vous (re)focaliser sur la matière. Je m'engage aussi à mettre en lumière l'essentiel. En contrepartie, j’attends de vous que vous participiez en partageant vos idées et connaissances.
    \item Après avoir couvert un concept, vous allez réaliser des exercices en rapport avec celui-ci. On n’apprend pas seulement à rouler en lisant le Code de la route. Il faut s'entrainer. C'est pourquoi je vous apprends à combiner et manipuler les concepts que nous avons vus au travers d’exercices. Ces derniers couvrent progressivement la matière du cours. Ils vous entrainent de manière simple à la réalisation de tâches plus complexes, car même si un exercice peut paraitre facile l'idée qu'il y a derrière le concept restera vraie même pour les exercices plus récapitulatifs. Ces défis sont à réaliser en classe ou parfois en pleine autonomie chez vous. Ils feront l'objet d'une correction collective en classe. Dans tous les cas, toutes les solutions seront disponibles.
    \item À certains moments, vous allez devoir rechercher individuellement ou collectivement un concept à partir d'un problème que je vous pose.
    \item Création, à domicile et individuellement, durant les différentes semaines de cours, d'un exercice récapitulatif\footnote{Il s’agit là d’un jeu que vous aurez choisi (voir la section \ref{eval_formative} sur l’évaluation formative).}. Ici, il n'est plus question de s'entrainer sur quelques points de matière, mais d'en combiner beaucoup. Il s’agit là d’une occasion de vous entrainer à l’examen et de revoir les points de matière du cours. Je vous encourage, au fur et à mesure que nous voyons les concepts théoriques, de les mettre en pratique dans votre jeu.\\
\end{itemize}

Parmi nos outils de travail, j'apprécie particulièrement Moodle. Il nous rend bien des services. Je m'en sers pour publier, au fur et à mesure et avant chaque début de cours, la matière que nous allons voir ensemble. Cela permet de fournir une structure claire et logique qui vous aide à comprendre. Cela rend le cours plus engageant et agréable, car vous savez ce qui va se passer. Vous pourrez vous y préparer. Aussi cela peut vous servir si vous avez une absence. Ainsi vous pouvez, avec l'aide des autres étudiants, vous remettre en ordre. Bien entendu, j'y publie tous les supports de cours et ressources utilisés. Enfin, j’y mets en place un forum. Ce dernier vous permet de poser des questions sur les exercices qui vous posent problème en attendant le prochain cours. L'idée n'est pas de s'échanger la solution, mais de discuter de vos difficultés en vous aidant les uns les autres. Bien entendu, j'interviendrai toujours pour valider ou complémenter vos idées.\\

Cela fait maintenant plusieurs années que je mets en place ce mode de fonctionnement. Et je constate qu'il est encore plus efficace si vous avez une approche \emph{proactive}. Ce qui signifie prendre des initiatives et donc agir de manière anticipée plutôt que de simplement réagir aux évènements. Cela peut se manifester de différentes manières, par exemple en prenant l'initiative d'apprendre les concepts progressivement plutôt que d'attendre l'ultime moment. Être proactif peut être particulièrement utile pour vous, cela vous permet de prendre en main votre propre apprentissage et de mieux gérer votre temps et vos responsabilités.

\clearpage
\subsection{Justifications}

\citet{perrenoud1992differenciation} dit~: «~différencier, c’est organiser les interactions et les activités de sorte que chaque élève soit constamment ou du moins très souvent confronté aux situations didactiques les plus fécondes pour lui~». J’essaye donc, face à la diversité mathétique, d’apporter une polyvalence didactique. J'utilise la méthode des évènements d’apprentissage\cite{leclercq2008modele} pour varier mes méthodes d'enseignants~:
\begin{itemize}
    \item La transmission~: me permet de couvrir de façon structurée un grand nombre de concepts tout en contrôlant le rythme de l'apprentissage et de m'assurer que tous les étudiants suivent.
    \item Le débat~: il se présente de deux manières différentes. D'une part, pendant le cours (synchrone) quand je fais réagir les étudiants sur un concept. Mais aussi de manière asynchrone avec le forum. Puisqu'ils vont relire les propositions des autres pour y réagir et débattre. J'y interviens comme modérateur.
    \item L'exercisation : aussi simple que cela puisse paraitre, je pense que pour devenir un bon programmeur il faut programmer. Il faut s'entrainer et faire des exercices jusqu'à développer certains réflexes. C'est pourquoi les concepts théoriques que j'aborde avec eux n'ont de sens que s'ils les appliquent concrètement.
    \item L'exploration~: je le mets régulièrement en place pour introduire un nouveau concept. Je n'aime pas leur donner directement la solution. J'essaye de leur faire ressentir l'intérêt du concept au travers d'un problème. La solution à ce problème est le nouveau concept à découvrir. Dans un deuxième temps, je repasse sur de la transmission classique pour fixer le concept qu'ils ont exploré~;
    \item La création~: les étudiants ont un exercice récapitulatif à réaliser. Ce projet regroupe presque tous les concepts vus en classe. Pour ce projet, je guide mes étudiants. Je les aide en leur donnant du feedback individuel. J'en reparlerai dans la section \ref{eval_formative} sur l'évaluation formative.
    \item La métacognition~: quand je corrige les exercices avec les étudiants, je mets en lumière les stratégies qu'il faut mettre en place pour résoudre facilement l'exercice. Souvent, je demande, aux étudiants qui ont bien réussi l'exercice, d'expliquer aux autres comment ils ont procédé. J'espère ainsi leur donner une occasion de réfléchir à leur manière de penser et résoudre un exercice. Selon \citet{dunlosky2013improving},la métacognition peut aider les étudiants à réussir en leur permettant de planifier, surveiller et réguler leur propre processus de pensée et d'apprentissage.
\end{itemize}
\begin{figure}[H]
    \begin{center}
        \includegraphics[width=0.9\textwidth]{figures/EAEs.eps}
        \caption{Représentation atomique de mes évènements dominants~\cite{leclercq2008modele}}
    \end{center}
\end{figure}
Le modèle de ~\citet{perrenoud1992differenciation} me permet de décrire de manière précise mes méthodes d'enseignement. Ce qui est très intéressant d'observer, c'est que je ne couvre pas tous les évènements. Par exemple, je pourrais davantage voir comment mettre en place de l'expérimentation et ce que cela apporterait aux étudiants. Par exemple, je pourrais leur fournir des programmes partiels et eux devraient essayer de les compléter avec un nouveau concept, ce serait une forme d’exploration. Je pense aussi que l'imitation pourrait avoir une place plus importante. Cela intervient de manière ponctuelle quand je leur montre comment je configure certains outils, mais ce n'est pas assez représentatif que pour en parler ici.

J'aimerais apporter une précision sur \emph{l’exercisation} qui occupe une part importante du cours. De prime à bord on pourrait se dire que cet évènement peut être ennuyeux et manquer de sens pour les étudiants s'ils ne comprennent pas pourquoi ils sont censés le faire. C'est pourquoi je commence toujours par contextualiser le concept. Je présente «~quand~» et «~pourquoi~» ce concept est utile. Puis, je le décontextualise au travers d'exercices simples en retirant de la complexité liée au contexte d'utilisation habituel afin de l'enseigner simplement pour maintenir le sentiment de faisabilité. Enfin, au fil des exercices, je recontextualise en complexifiant le contexte d'utilisation de sorte à aller vers l’utilisation authentique de ce dernier. La complexité majeure de cette pratique étant de trouver le bon équilibre entre la décontextualisation et la contextualisation.\\

Selon \citet{patry2004etayage}, l'étayage est un soutien apporté à l'apprenant pour lui permettre de progresser dans ses apprentissages et atteindre ses objectifs. Ces auteurs proposent également une typologie de l'étayage en fonction de son degré de structuration et de l'implication de l'enseignant. Selon eux, il existe quatre types d'étayage:

\begin{enumerate}
    \item L'étayage minimal: il s'agit d'un soutien minimal de l'enseignant, qui se limite à fournir des informations sur la tâche et vérifier que l'apprenant comprend bien les consignes. L'enseignant n'intervient pas dans la réalisation de la tâche elle-même.
    \item L'étayage structuré: l'enseignant fournit des consignes détaillées et précises sur la manière de réaliser la tâche, et accompagne l'apprenant dans sa réalisation. Il peut également offrir des exemples ou des modèles à suivre.
    \item L'étayage collaboratif: l'enseignant et l'apprenant travaillent ensemble de manière collaborative pour réaliser la tâche. L'enseignant apporte son expertise et ses connaissances, tandis que l'apprenant apporte son savoir-faire et son expérience.
    \item L'étayage coopératif: l'enseignant et l'apprenant travaillent en équipe pour réaliser la tâche, chacun apportant ses compétences et ses connaissances. L'enseignant peut également jouer le rôle de facilitateur pour aider l'équipe à travailler de manière efficace.
\end{enumerate}

Quand je réalise les exercices avec mes étudiants, je vise le niveau 3 et 4. C'est-à-dire que je donne d'abord des consignes claires et structurées sur ce qui est attendu, en leur donnant des pistes concrètes. Il s'agit là plutôt des deux premiers niveaux. Ensuite, quand chacun a eu l'occasion d'essayer l'exercice, je passe au troisième et quatrième niveau. À ce moment précis, je donne le câble de projection à un étudiant (pour qu’il puisse projeter son écran) et nous réalisons/programmons tous ensemble l'exercice. L'objectif ici est de mettre en commun toutes les idées de sorte à aboutir à une solution. C'est donc les étudiants tous ensemble qui apportent la solution et moi j'agis comme un \emph{facilitateur} et \emph{organisateur}. Voici quelques problèmes que j'ai identifiés :
\begin{itemize}
    \item L'étayage entraine une dépendance de l'apprenant vis-à-vis de l’enseignant, ce qui peut nuire à son autonomie. J'essaye donc, au fil des semaines, de les conditionner de sorte qu'ils apprennent les bons réflexes sans plus avoir besoin de moi.
    \item Le temps que cela prend. Je ne peux que corriger quelques exercices par séances. Cela demande beaucoup de ressources.
    \item Certains étudiants ont tendance à se reposer sur les propositions des autres. Et donc ils sont moins impliqués. Cela dit, même si l’écran d’un seul étudiant est projeté, les autres étudiants codent aussi sur leur machine, sinon ils n'ont pas la solution (faite en classe).\\
\end{itemize}

Je suis convaincu que cette pratique apporte un intérêt certain au début, mais il faut aussi développer leur autonomie. D'ailleurs, \citet{auger2010desetayage} ont étudié le désétayage. Il consiste à diminuer progressivement le soutien apporté à l'apprenant pour lui permettre de devenir plus autonome dans ses apprentissages. Les auteurs mettent en avant les bénéfices du désétayage pour l'apprenant, notamment en développant son autonomie. J'ai identifié dans cet article 5 conseils pratiques. J'ai mis en regard de chacune des recommandations un degré d'aboutissement personnel avec une justification :

\begin{itemize}
    \item Faire le point sur les compétences de l'apprenant et évaluer son niveau de maitrise de la tâche avant de débuter le processus de désétayage. (40\%) Étant donné que mon désétayage est progressif, je n'ai pas mis en place un moment où l'étudiant peut officiellement, mesurer son degré de maitrise. Même si à chaque fois qu'il fait un exercice en autonomie, il devrait se rendre compte de sa maitrise des concepts. Il serait donc intéressant de mettre en place une évaluation formative supplémentaire pour qu'ils aient un feedback officiel.
    \item Proposer des tâches d'une difficulté croissante à l'apprenant pour lui permettre de progresser et de développer son autonomie. (90\%) Clairement, mes exercices sont construits avec une difficulté croissante.
    \item Encourager l'apprenant à poser des questions et à demander de l'aide lorsqu'il en a besoin, tout en le faisant travailler de manière autonome le plus possible.(90\%) Je suis très attentif à la participation des étudiants. Je pense ressentir quand un étudiant ne comprend plus. J'essaye donc de le faire réagir pour éclairer le concept en partant de ce qu'il sait déjà.
    \item Favoriser la collaboration et le travail en équipe pour encourager l'apprenant à s'appuyer sur les compétences et les connaissances de ses pairs.(65\%) Dans mon cours, ils n'ont pas vraiment de projet d'équipe, mais j'essaye de les faire travailler ensemble lors de la correction.
    \item Favoriser l'apprentissage par projet, qui permet à l'apprenant de s'impliquer dans un projet concret et de développer ses compétences de manière autoacpnome.(80\%) D'où l'intérêt du projet personnel.
\end{itemize}

Cet article est donc particulièrement intéressant puisque j'ai identifié des pistes d’amélioration concrètes à mettre en place pour l'année prochaine.\\

Comme on peut le lire dans la partie pour les étudiants, je les rassure sur l'idée que derrière les exercices d'apparence simple se cachent des invariants qu'ils doivent réappliquer pour les exercices plus intégratifs. D'ailleurs, \citet{piaget1970structures} en parle dans ses travaux sur le développement cognitif de l'enfant. Selon lui les \emph{invariants opératoires} sont des structures mentales qui permettent à l'enfant de traiter l'information de manière cohérente et de résoudre des problèmes \cite{piaget1970structures}.

Il identifie deux types d'invariants opératoires : les \emph{invariants formels} et les \emph{invariants opérationnels concrets}. Les invariants formels sont des structures mentales abstraites qui permettent à l'enfant de traiter des informations symboliques, comme les nombres ou les lettres. Les invariants opérationnels concrets sont des structures mentales qui permettent à l'enfant de traiter des informations concrètes, comme les objets ou les actions.

Toujours selon \citet{piaget1970structures}, l'acquisition des invariants opératoires est un processus progressif qui se déroule en plusieurs étapes. L'enfant commence par acquérir des invariants opérationnels concrets, puis il passe aux invariants formels. Cette progression permet à l'enfant de développer de plus en plus de capacités de raisonnement et de logique.

Dans mon cours ces invariants opératoires sont utilisés pour les  aider à comprendre les concepts abstraits puis à résoudre les tâches plus complètes.\\

«~Make learning visible~» \footnote{«~The 'visible' aspect also refers to making teaching visible to the student, such that they learn to become their own teachers.~»\cite{hattie2012visible}}\cite{hattie2012visible}. Le projet est aussi une occasion, pour les apprenants (et moi-même), de se rendre compte des savoirs qu'ils acquièrent. Ils voient bien qu'au fur et à mesure que la matière est vue, ils peuvent avancer dans la réalisation de leur propre jeu.\\
Étant donné que la motivation fait partie de ma philosophie et que l’un des ingrédients de la motivation c’est d’apporter de la valeur aux connaissances\cite{viau1994motivation}, ce projet me permet de rendre concrets mes enseignements au travers de besoins issus de situations authentiques.\\



\clearpage
\section{Évaluation des apprentissages}

\subsection{L’évaluation formative}

\label{eval_formative}
Cette évaluation formative a pour but de vous \emph{entrainer} sans pour autant vous pénaliser. Ce qui m'importe ici, c'est que vous \emph{appreniez} et \emph{développiez} les aptitudes dont vous avez besoin pour votre futur métier. Je cherche donc ici à vous donner l'occasion de vivre l'examen pratique pour mieux vous y préparer et surtout identifier ce que vous avez déjà compris et ce que vous devez encore travailler.
\begin{itemize}
    \item Lors de l'avant-dernière séance de cours~: vous réaliserez individuellement un examen des années précédentes en classe dans les mêmes conditions. C'est à dire en 4 heures et à court ouvert. Nous consacrerons la dernière séance à sa correction collective où chacun corrige individuellement sa copie d'examen.
\end{itemize}

\subsection{L’évaluation certificative}

\label{eval_certificative}
L’évaluation certificative s'organise en 2~temps~:
\begin{itemize}
    \item Vous devez rendre, le jour de l'examen, votre projet de jeu personnel que vous aurez développé individuellement chez vous pendant les différentes semaines de cours. Les consignes vous seront communiquées au premier cours. Vous devez donc gérer votre temps pour ce projet qui compte pour 20~\% de la cote finale. Cela vous permet de mettre en pratique les concepts et les techniques apprises au cours dans un contexte concret et significatif. Cela est aussi une occasion de vous démarquer et de montrer tout ce que vous avez appris. Enfin puisqu'il s'agit là d'un projet personnel que vous pouvez mettre dans votre portfolio, c'est aussi une occasion de vous faire également remarquer par un futur employeur. Je tiens aussi à vous rassurer que vous avez de temps pour réaliser cette tâche. Ce cours compte pour 6 crédits, soit un volume horaire d'environ 165 heures. Dont seulement 60 heures de cours. Après avoir interrogé vos prédécesseurs, j'estime qu'il vous faut raisonnablement une quinzaine d'heures, soit environ 1 heure de travail par semaine de cours.
    \item L'examen pratique consiste à programmer un jeu à partir d'un énoncé\footnote{Je vous rappelle que tous les énoncés des années précédentes sont disponibles sur l'\href{https://github.com/tecg-mmi}{organisation GitHub} officielle du cours. (github.com/tecg-mmi)} que vous découvrirez le jour même. J'ai pour objectif de vous mettre, le plus possible, dans des situations de travail réalistes. C'est-à-dire celles que vous pourriez possiblement rencontrer lors de votre premier emploi. C'est donc pour cela que l'examen se fait sur vos machines à cours ouvert avec la documentation officielle ainsi que vos productions personnelles. Notez cependant que la limite à ne pas franchir est la communication avec autrui\footnote{"Autrui" désigne une personne ou un groupe de personnes différentes de soi-même.}. Vous disposez de 4~heures pour réaliser cet examen en classe. Ce travail compte pour 80~\% de la cote finale. C'est ainsi que je peux m'assurer votre maitrise des concepts clés en les mettant en pratique dans un cas concret de manière autonome.
\end{itemize}
Lors de la première séance de cours, je vous présenterai l'énoncé du dernier examen avec sa grille d'évaluation. Elle se construit toujours de la même manière. J'attribue aux fonctionnalités du jeu un degré de complexité. Ensuite, je mesure l'aboutissement des différentes fonctionnalités dans votre proposition. Ces fonctionnalités sont clairement mentionnées dans l'énoncé de l'examen et sont classées par ordre de complexité. Le nombre de points maximum attribué à chaque fonctionnalité est également mentionné. \\
Pour la seconde session, les conditions sont exactement les mêmes. Vous devez venir avec votre jeu personnel\footnote{Vous pouvez améliorer ou reprendre votre jeu si vous l’avez déjà réussi.} et passer l'examen pratique. Je vous encourage, si jamais vous n'avez pas réussi en première session, de venir me voir à la journée rencontre avec vos professeurs pour voir votre copie d'examen afin que je puisse vous donner un retour personnalisé sur les concepts que vous devez retravailler.
\clearpage
\subsection{Justifications}
\label{evaluation_des_apprentissages_justifications}

Tout d'abord, je tiens à apporter une précision sur la partie pour les étudiants. Contrairement à ma première version, j'ai choisi de ne plus parler ici du test qu'ils font en début d'année, car je n'évalue pas la matière du cours, mais un corequis. C'est une technique que je mets en place pour vérifier leur acquis, mais elle n'est pas en lien avec l'évaluation des apprentissages de ce cours-ci.\\

«~L’émission de feedbacks est souvent considérée comme un élément clé pour renforcer la motivation et soutenir la réussite des élèves.~»\cite{georges2011feedbacks}. Cependant, j'ai identifié, au travers de mes lectures, des conditions pour qu'il soit pleinement profitable à l'étudiant :

\begin{itemize}
    \item Positif et constructif : selon \citet{hattie2007power}, le feedback doit être orienté vers les points forts de l'étudiant et doit proposer des pistes de progrès plutôt que de se concentrer sur les faiblesses. C'est assez compliqué pour moi lors de la correction de mettre en lumière les points forts de manière individuelle, car je ne corrige pas leur copie. En revanche, après avoir corrigé une partie en plénière, je demande régulièrement qui a su réaliser cette tâche et j’encourage ces étudiants. Pour les autres, qui n'ont pas su faire la tache, je m'assure qu'ils ont compris en leur demandant de reformuler, puis les rassurant qu’ils savent maintenant et peuvent rectifier le tir pour la prochaine fois.
    \item Ciblé et spécifique : Selon \citet{black1998assessment}, le feedback doit être précis et se concentrer sur des aspects ciblés de la performance de l'étudiant, plutôt que de donner une évaluation générale. Ici, le feedback est très spécifique, mais surtout indirect. Il ne s'agit pas d'un feedback général que je donne qui est valable pour tout le monde, mais d'une prise de conscience personnelle. L'étudiant se rend compte des points qu'il doit travailler. Pour rencontrer le caractère précis dont parlent \cite{black1998assessment}, il faudrait que je formule moi-même un feedback personnalisé pour chaque étudiant. C'est une piste que j'envisage pour l'année prochaine.
    \item Temporellement proche : selon \citet{kluger1996effects}, le feedback doit être donné de manière opportune, de préférence immédiatement après la performance de l'étudiant, afin de maximiser son impact sur l'apprentissage. Techniquement, il n'est pas possible d'envisager la correction directement après, car je n'ai que 4 heures de cours par semaine. La correction se fait directement la semaine qui suit. Je pourrais, pour les prochaines années, voir s'il ne serait pas intéressant pour les étudiants d'organiser une journée de cours avec 4 heures d'examen et puis après le temps de midi 2 heures de correction.
\end{itemize}

De plus, cette correction permet de réfléchir ensemble aux stratégies qu'il faut mettre en place pour réussir l'examen. \citet{winne1998studying} parlent de \emph{l'attention métacognitive}, cela consiste à être conscient de son propre processus de pensée et à être capable de se concentrer sur les informations importantes. Toujours selon eux, cela peut aider les étudiants à mieux comprendre et à mieux retenir ce qu'ils apprennent. De plus, je cherche au travers de cette correction d'ajuster leur manière  de procéder, en leur montrant ce qui est efficace et efficient. Mon intention est qu'ils puissent s'autocorriger et trouver des solutions lorsqu'ils rencontrent ces mêmes problèmes. \citet{zimmerman1986becoming} ont démontré l'intérêt de cette pratique dans leur travail, ils parlent de \emph{la régulation métacognitive}.
D'ailleurs chaque année je désigne un «~sécrétaire~» pour la séance de correction. Il aura pour mission de prendre note de toutes les astuces que nous avons déterminées ensemble afin qu’ils puissent les consulter plus tard. Bien entendu, je prends soin, dans la rédaction de l'énoncé, d'être constant. À vrai dire, je réutilise un gabarit de base pour rédiger l'énoncé d’examen afin qu’ils ne soient pas surpris par la forme le jour de l'examen.


\clearpage
\section{Alignement pédagogique}

Je suis impatient de travailler avec vous pour développer vos compétences en programmation. Je suis convaincu que, grâce à votre motivation et à votre engagement, vous serez en mesure de devenir de véritables experts en JavaScript.

Je tiens à vous assurer que j'ai particulièrement travaillé «~la cohérence des objectifs~», «~les méthodes utilisées~» et leur «~évaluation~». Cette cohérence vous permettra d'atteindre l'objectif d'apprentissage de manière plus efficace. Mes visées d'apprentissage sont clairement définies et sont en lien avec les compétences que vous devez développer pour devenir de bons programmeurs en JavaScript. J'ai choisi des méthodes d'enseignement variées qui vous permettront de mettre en pratique et de comprendre les concepts dont vous avez besoin. De plus, nous allons avoir des moments d'échanges où vous pourrez poser vos questions et partager vos idées avec moi et avec vos camarades de classe. Enfin, j'ai prévu des évaluations qui me permettent de mesurer l'acquisition de vos compétences de manière juste.

En suivant ce cours, vous allez donc travailler sur les concepts fondamentaux de la programmation en JavaScript et vous allez mettre en pratique ce que vous avez appris grâce à beaucoup d'exercices et un projet personnel. Je vais vous donner des exemples concrets pour vous aider à mieux les comprendre.

Je suis convaincu qu’ensemble et avec votre détermination, vous serez en mesure de devenir de véritables programmeurs web. Je vous encourage donc à poser toutes les questions que vous avez, à participer activement aux discussions en classe et à travailler pour réussir votre projet personnel et l’examen.

\subsection{Justifications}
\begin{table}[h]
    \begin{tabular}{|p{0.30\textwidth}|p{0.30\textwidth}|p{0.30\textwidth}|}
        \hline
        \multicolumn{1}{|c|}{\textbf{Objectifs d'apprentissage}} & \multicolumn{1}{c|}{\textbf{Méthodes d'enseignement}} & \multicolumn{1}{c|}{\textbf{Évaluation des apprentissages}}
        \\ \hline\hline
        Connaitre les concepts                                   & Transmission  - exploration                           & \multirow{2}{*}{\begin{tabular}[c]{@{}l@{}}Ils ne doivent pas les réciter mais\\ les mettre en application dans \\ le projet personnel et l'examen\end{tabular}} \\ \cline{1-2}
        Comprendre ces concepts                                  & Exercisation - débat - exploration                    &                                                                                                                                                                  \\ \hline
        Identifier les concepts                                  & Exercisation (exercices + examen formatif)            & \multirow{3}{*}{\begin{tabular}[c]{@{}l@{}}* Projet personnel\\ * Examen pratique\end{tabular}}                                                                  \\ \cline{1-2}
        Savoir combiner différents concepts                      & Exercisation (exercices + examen formatif)            &                                                                                                                                                                  \\ \cline{1-2}
        Savoir utiliser les outils de développement vus en cours & Imitation - transmission                              &                                                                                                                                                                  \\ \hline
    \end{tabular}
\end{table}

Ce tableau illustre que j'entraine mes étudiants aux objectifs du cours et que je les y évalue, mais il montre également que «~le projet personnel~» et «~l'examen pratique~» évaluent les mêmes visées d'apprentissages. Cela signifie que, bien que ces évaluations soient différentes en termes de format et de durée, elles ont toutes deux pour objectif final de mesurer la compréhension et la mise en pratique des concepts.

Il est également important de noter que ces évaluations sont nécessaires pour leur apprentissage. C'est pourquoi j'ai pensé qu'il serait intéressant, afin de soulager mes étudiants et de leur offrir une expérience d'apprentissage plus enrichissante, d'évaluer la compréhension des concepts lors d'un examen qui serait alors plus court et de mesurer la mise en pratique et la combinaison des concepts dans le projet personnel. Cela permettrait de varier les évaluations et de donner aux étudiants l'opportunité de mettre en pratique ce qu'ils ont appris de manière approfondie et précise.

\clearpage

\section{Modalités organisationnelles}

Vous pouvez me contacter de différentes manières :
\begin{enumerate}
    \item Si vous avez des questions techniques liées à une incompréhension et/ou un problème avec un exercice, je vous demanderai de la poser sur le forum officiel du cours sur Moodle.
    \item Pour toutes autres communications, je vous demande de me contacter par mail\\ \href{mailto:daniel.schreurs@hepl.be}{daniel.schreurs@hepl.be}.
    \item Si vous avez des informations urgentes à me faire parvenir, vous pouvez me joindre directement via Teams.
\end{enumerate}
\section{Ressources}
\begin{itemize}
    \item Moodle~: toutes les ressources nécessaires pour le cours sont référencées sur la page officielle du cours. Je choisis cette plateforme, car il s’agit de l’outil officiel recommandé et utilisé au sein de la HEPL. Vous y trouverez des sections avec~:
          \begin{itemize}
              \item Le déroulement de toutes les séances de cours~: avec la matière couverte et les liens vers les ressources utilisées lors de cette séance. Ainsi vous avez une trace écrite de ce que nous avons vu à chaque séance.
              \item La fiche ECTS :~présentée au premier cours pour que vous poussiez la relire.
              \item Les notes de cours au format PDF et Microsoft PowerPoint\footnote{Je vous donne également ce format, car je sais que certains étudiants aiment prendre note dans la partie "note de l'intervenant".}, ainsi vous pouvez les télécharger avant le cours et les annoter pendant que je donne les explications.
              \item Toutes les ressources auxiliaires~: que j'utilise en classe.
          \end{itemize}
    \item Logiciels nécessaires~: Il est indispensable d'avoir un environnement de travail informatique opérationnel. Nous utiliserons la même configuration de machine que pour le cours de «~Développement Côté Client~». Vous pouvez retrouver toutes les installations à faire \href{https://github.com/tecg-dcc/js-ressources#environnement-de-travail}{ici}\footnote{github.com/tecg-dcc/js-ressources}.
\end{itemize}
\clearpage
\section{Réflexion personnelle}

Quand on m'a présenté les fiches ECTS à la HEPL, je n'avais pas ressenti l'intérêt pour mes étudiants. Je pensais qu'il s'agissait d'une formalité administrative. À l'issue de cette rédaction, je suis convaincu qu'ils servent à motiver mes étudiants et à créer un climat de confiance mutuel en communiquant avec eux de manière claire et transparente. De toute évidence, il s'agit là d'un outil stratégique que je souhaite mettre à profil. Je me souviens, quand j'étais un enfant, je ne supportais pas qu'on m'impose de manière arbitraire des choix. Maintenant, j'ai bien compris pourquoi il est nécessaire de faire comprendre les enjeux et les motivations d'une décision afin d’évaluer si elle est justifiée. C'est pour ces raisons que j'ai accordé un soin tout particulier à justifier auprès de mes étudiants le pourquoi~et le pourquoi du comment. \\
Enfin, cet écrit a été pour moi une occasion de réfléchir sur «~ce que je fais~» et «~pourquoi je le fais~». Et plus précisément, une occasion de confronter mes pratiques d'enseignants à des modèles existants. Cela m'a permis d’identifier les aspects de ma pratique qui sont efficaces et ceux qui pourraient être améliorés. Cela m'a aussi permis de faire naitre des nouvelles idées et techniques. Je suis fier d'avoir réalisé ce document et suis convaincu qu'il me prépare au portfolio d'intégration.

\clearpage
\section{Bibliographie}
\printbibliography[heading=none]

%\bibliography{citations}
%\bibliographystyle{apalike-fr}





\end{document}
